\documentclass[11pt]{article}

\usepackage{minted}
\usepackage{hyperref}
\usepackage{datetime}
\usepackage{datenumber}
\usepackage{advdate}
\usepackage{amsmath}
\usepackage[super]{nth}
\usepackage[margin=0.75in]{geometry}

\parindent 0pt \parskip 6pt

\newcommand\todo[1]{\textbf{TODO(kc506): #1}}
\newcommand\haskell[1]{\mintinline{haskell}{#1}}
\newcommand\monospace[1]{\mintinline{text}{#1}}

\begin{document}

\thispagestyle{empty}

\centerline{\large An Optimising Compiler from Haskell to Java Bytecode}
\centerline{\large Progress Report}
\vspace{0.2in}

\centerline{Keith Collister, kc506}
\centerline{Robinson College}
\centerline{\today}

\vspace{0.2in}

\ThisYear{2019}\ThisMonth{2}\ThisDay{4}
% Specify date format like "15th Oct"
\newdateformat{datefmt}{\nth{\THEDAY} \shortmonthname[\THEMONTH]}

% Force advdate commands to change the global "today" instance, not the local one
\makeatletter
\renewcommand\AdvanceDate[1][\@ne]{\global\advance\day#1 \FixDate}
\renewcommand\FixDate{%
  \FixMonth \is@LeapYear
  \l@@p \global\ifnum\day<1 \Pr@vD@y \repeat
  \l@@p \M@s\m@sic \global\ifnum\day>\M@s \N@xtD@y \repeat
}
\renewcommand\FixMonth{%
  \L@@p \global\ifnum\month<1 \global\advance\year\m@ne \global\advance\month12 \is@LeapYear \repeat
  \L@@p \global\ifnum\month>12 \global\advance\year\@ne \global\advance\month-12 \is@LeapYear \repeat}
\def\Pr@vD@y{%
  \global\ifnum\day<-366
    \global\ifnum\month>2
      \global\advance\day\r@k \global\advance\year\m@ne \is@LeapYear
    \else
      \global\advance\year\m@ne \is@LeapYear \advance\day\r@k
    \fi
  \else
    \global\advance\month\m@ne \FixMonth
    \global\advance\day\m@sic
  \fi}
\def\N@xtD@y{%
  \global\ifnum\day>366
    \global\ifnum\month>2
      \global\advance\year\@ne \is@LeapYear \global\advance\day-\r@k
    \else
      \global\advance\day-\r@k \global\advance\year\@ne \is@LeapYear
    \fi
  \else
    \global\advance\day-\M@s \global\advance\month\@ne \FixMonth
  \fi}
\makeatother

% Output eg. boldface "15th Oct - 29th Oct" when called.
\newcommand{\daterange}[1]{%
    \textbf{\datefmt\today}
    \textbf{--}
    \AdvanceDate[#1]\relax
    \AdvanceDate[-1]\relax
    \textbf{\datefmt\today}
    \AdvanceDate[1]\relax
}

\begin{tabular}[t]{@{}l}
{\bf Project Supervisor:} Dr.\ Timothy Jones\\[3mm]
{\bf Director of Studies:} Prof.\ Alan Mycroft
\end{tabular}
\hfill
\begin{tabular}[t]{@{}l @{}l}
{\bf Overseers: } & Prof.\ Simone Teufel \\[3mm]
& Dr.\ Andrew Rice
\end{tabular}

\vspace{0.3in}

\large{\bf Success Criteria}

The success criteria established in the project proposal were to support translating a basic subset of Haskell to lazily
evaluating Java bytecode, performing simple optimisations.

My project currently does not quite meet those criteria: the compiler translates Java bytecode from a medium-sized
subset of Haskell source code into executable bytecode, but it currently evaluates `more strictly' than intended, and
one core language feature (typeclass instances) currently has critical bugs when compiled to bytecode. No optimisations
have been implemented so far.

The main cause of the delay was unexpected dependencies between different source language features. To define simple
operations like addition, it is necessary to implement functions, datatypes, and typeclasses: this forced me to
implement some of these features sooner than planned (during Michaelmas rather than at the start of Lent term). By
implementing these to support simple operations, some planned features were delivered `for free' earlier than planned
(eg. lists, simply defined as \haskell{data [] a = [] | a:[a]} after implementing support for datatypes).

Another cause for delay was a period of unexpectedly high workload due to the Cloud Computing Unit of Assessment at the
end of Michaelmas term, which halted development for two weeks.

\large{\bf Current Progress}

The compiler contains 4 coarse stages:

\begin{enumerate}
\item
{
    Frontend: The Haskell source is parsed using an external library, which I have slightly modified to allow parsing
    built-in data constructors such as \haskell{(,)} and \haskell{[]}. Variables in the Haskell source are renamed to
    eliminate shadowing, reducing complexity in subsequent stages. Dependency analysis is performed on the source
    declarations: recursive relations are inferred and the order of compilation of declarations is decided.
}
\item
{
    Type Inference: Polymorphic overloaded type signatures (eg. \(\forall \alpha.\;\text{Num}\;\alpha \Rightarrow \alpha
    \rightarrow \alpha \rightarrow \alpha\)) are inferred from the source declarations using the results of the
    dependency analysis. The Haskell AST is explicitly annotated with type information, and any user-provided type
    signatures are checked against the inferred signatures.
}
\item
{
    Middle-end: Overloaded functions (eg. \haskell{(+) :: Num a => a -> a -> a}) are deoverloaded into
    `dictionary-passing' equivalents (\haskell{(+) :: Num a -> a -> a -> a}). Typeclasses are replaced with datatype and
    function declarations, and typeclass instances are replaced with variable declarations. These are all implemented as
    Haskell source-to-source transformations.

    The Haskell AST is lowered through two intermediate languages (ILs) and into Administrative Normal Form. The first
    IL provides a simple language that optimisation passes can easily process, and the second IL makes allocation and
    (lazy) evaluation explicit for easier code generation.
}
\item
{
    Code Generation: The final IL is transformed into Java Bytecode. An external library is used to convert a logical
    description of the bytecode to actual binary data, and I have extended the library to support various JVM features 
    required by the lowering (notably the \monospace{invokedynamic} and \monospace{lookupswitch} instructions).
}
\end{enumerate}

\large{\bf New Schedule}

Given that the project's schedule was effectively rearranged by implementing features sooner than expected, I've
included a new workplan below:

\begin{itemize}
\item
{
    \daterange{7}

    Debug currently failing bytecode for typeclass instances. Rewrite the translation of pattern matching to provide
    non-strict semantics.

    After this work package, the compiler will be able to transform a reasonably-sized subset of Haskell to executable
    Java Bytecode.
}
\item
{
    \daterange{21}

    Implement optimisations: currently, I intend to implement a peephole pass, unreachable code/procedure elimination,
    dead store elimination, lambda+let lifting, and if there is sufficient time, strictness analysis.
}
\end{itemize}

After the end of these two work packages, I should be only a week behind the original proposed schedule, and intend to
spend any remaining time fixing bugs and starting on my dissertation write-up.

\end{document}