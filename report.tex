\documentclass[11pt]{article}

\usepackage{minted}
\usepackage{hyperref}
\usepackage{datetime}
\usepackage{datenumber}
\usepackage{advdate}
\usepackage[super]{nth}
\usepackage[margin=0.75in]{geometry}

\parindent 0pt \parskip 6pt

\newcommand\todo[1]{\textbf{TODO(kc506): #1}}
\newcommand\haskell[1]{\mintinline{haskell}{#1}}
\newcommand\monospace[1]{\mintinline{text}{#1}}

\begin{document}

\thispagestyle{empty}

\centerline{\large An Optimising Compiler from Haskell to Java Bytecode}
\centerline{\large Progress Report}
\vspace{0.2in}

\centerline{Keith Collister, kc506}
\centerline{Robinson College}
\centerline{Foobar date}

\vspace{0.2in}

\ThisYear{2019}\ThisMonth{1}\ThisDay{21}
% Specify date format like "15th Oct"
\newdateformat{datefmt}{\nth{\THEDAY} \shortmonthname[\THEMONTH]}

% Force advdate commands to change the global "today" instance, not the local one
\makeatletter
\renewcommand\AdvanceDate[1][\@ne]{\global\advance\day#1 \FixDate}
\renewcommand\FixDate{%
  \FixMonth \is@LeapYear
  \l@@p \global\ifnum\day<1 \Pr@vD@y \repeat
  \l@@p \M@s\m@sic \global\ifnum\day>\M@s \N@xtD@y \repeat
}
\renewcommand\FixMonth{%
  \L@@p \global\ifnum\month<1 \global\advance\year\m@ne \global\advance\month12 \is@LeapYear \repeat
  \L@@p \global\ifnum\month>12 \global\advance\year\@ne \global\advance\month-12 \is@LeapYear \repeat}
\def\Pr@vD@y{%
  \global\ifnum\day<-366
    \global\ifnum\month>2
      \global\advance\day\r@k \global\advance\year\m@ne \is@LeapYear
    \else
      \global\advance\year\m@ne \is@LeapYear \advance\day\r@k
    \fi
  \else
    \global\advance\month\m@ne \FixMonth
    \global\advance\day\m@sic
  \fi}
\def\N@xtD@y{%
  \global\ifnum\day>366
    \global\ifnum\month>2
      \global\advance\year\@ne \is@LeapYear \global\advance\day-\r@k
    \else
      \global\advance\day-\r@k \global\advance\year\@ne \is@LeapYear
    \fi
  \else
    \global\advance\day-\M@s \global\advance\month\@ne \FixMonth
  \fi}
\makeatother

% Output eg. boldface "15th Oct - 29th Oct" when called.
\newcommand{\daterange}[1]{%
    \textbf{\datefmt\today}
    \textbf{--}
    \AdvanceDate[#1]\relax
    \AdvanceDate[-1]\relax
    \textbf{\datefmt\today}
    \AdvanceDate[1]\relax
}

\begin{tabular}[t]{@{}l}
{\bf Project Supervisor:} Dr. Timothy Jones \\[3mm]
{\bf Director of Studies:} Prof. Alan Mycroft
\end{tabular}
\hfill
\begin{tabular}[t]{@{}l @{}l}
{\bf Overseers: } & Prof. Simone Teufel \\[3mm]
& Dr. Andrew Rice
\end{tabular}

\vspace{0.3in}

\large{\bf Success Criteria}

My project currently does not meet the minimum success criteria established in the proposal: the compiler produces Java
bytecode from Haskell source code, but there are bugs that prevent the output from being executed. In addition, no
optimisations have yet been implemented.

The main cause of the delay is unexpected dependencies between different source language features. To define simple
operations like addition, it is necessary to implement functions, datatypes, and typeclasses: this forced me to
implement these features sooner than planned. Currently functions, datatypes, and typeclass definitions have been
implemented (with known bugs), and typeclass instance definitions are the remaining critical feature. After implementing
these, some planned features (eg. lists) are delivered `for free' (simply defined as \haskell{data [] a = [] | a:[a]}).

Another cause for delay was a period of unexpectedly high workload at the end of Michaelmas term, which stalled
development for two weeks.

\large{\bf Current Progress}

The compiler contains 4 coarse stages:

\begin{enumerate}
\item
{
    Frontend: The Haskell source is parsed using an external library. Variables in the Haskell source are renamed to
    eliminate shadowing, reducing complexity in subsequent stages. Dependency analysis is performed on the source
    declarations: recursive relations are inferred and the order of compilation of declarations is decided.
}
\item
{
    Typechecking: Type signatures are inferred from the source declarations using the results of the dependency
    analysis. The Haskell AST is explicitly annotated with type information, and any user-provided type signatures are
    checked against the inferred signatures.
}
\item
{
    Middle-end: Overloaded functions (eg. \haskell{(+) :: Num a => a -> a -> a}) are deoverloaded into
    `dictionary-passing' equivalents (\haskell{(+) :: Num a -> a -> a -> a}). Typeclasses are replaced with datatype
    and function declarations.

    The Haskell AST is lowered through two intermediate languages and into Administrative Normal Form. The first IL
    provides a simple language that optimisation passes can easily process, and the second IL makes (lazy) evaluation
    and allocation explicit for easier code generation.
}
\item
{
    Code Generation: The final IL is transformed into Java Bytecode. An external library is used to convert a logical
    description of the bytecode to actual binary data, but I have extended the library to support various JVM features 
    required by the lowering (notably the \monospace{invokedynamic} and \monospace{lookupswitch} instructions).
}
\end{enumerate}

\large{\bf New Schedule}

Given that the project's schedule was effectively rearranged by implementing features sooner than expected, I've
included a new workplan below:

\begin{itemize}
\item
{
    \daterange{14}

    Debug currently failing bytecode, and implement typeclass instances (allows for specifying datatypes to be instances
    of certain typeclasses). Ensure non-strict semantics are working as expected.

    After this work package, the compiler will be able to transform a reasonably-sized subset of Haskell to executable
    Java Bytecode.
}
\item
{
    \daterange{21}

    Implement optimisations: currently, I intend to implement a peephole pass, dead store elimination, lambda+let
    lifting, and strictness analysis.
}
\end{itemize}

After the end of these two work packages, I should be on track with the original proposed schedule, and intend to spend
the remaining time fixing bugs and starting on my dissertation write-up.

\end{document}